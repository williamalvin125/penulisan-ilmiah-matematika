\begin{center}
  \large\textbf{ABSTRACT}
\end{center}

\addcontentsline{toc}{chapter}{ABSTRACT}

\vspace{2ex}

\begingroup
% Menghilangkan padding
\setlength{\tabcolsep}{0pt}

% Ubah paragraf berikut dengan abstrak dari tugas akhir dalam Bahasa Inggris
This study formulates and analyzes a mathematical model for the spread of drug abuse in Indonesia by incorporating rehabilitation mechanisms (inpatient and outpatient), relapse, drug-related mortality, and transmission rates through social interactions; the model is a modification of the framework developed by Mushanyu et al. and is adapted to represent transitions between compartments as well as the effects of rehabilitation capacity. By defining variables as the proportions of susceptible individuals, users (with/without rehabilitation), and rehabilitation patients, the study sets model parameters, carries out numerical simulations, and performs equilibrium-point analysis and linearization to determine system stability. Results indicate the existence of a drug-free equilibrium that represents the ideal state without users; however, linear analysis shows instability around this point. Numerical simulations further suggest that increasing the recruitment proportion into rehabilitation can suppress transmission rates and reduce user prevalence, meaning that policies to expand access to and recruitment for rehabilitation are likely effective for mitigation. These findings underscore the importance of adequate rehabilitation programs and social-intervention policies to control the dynamics of drug abuse.

% Ubah kata-kata berikut dengan kata kunci dari tugas akhir dalam Bahasa Inggris
\bigskip
\noindent\textbf{Keywords:} mathematical model, drug abuse, rehabilitation (inpatient/outpatient), stability, numerical simulation, equilibrium point.
