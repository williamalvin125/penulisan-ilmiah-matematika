\begin{center}
  \large\textbf{ABSTRAK}
\end{center}

\addcontentsline{toc}{chapter}{ABSTRAK}

\vspace{2ex}

\begingroup
% Menghilangkan padding
\setlength{\tabcolsep}{0pt}


% Ubah paragraf berikut dengan abstrak dari tugas akhir
Topik ini menyusun dan menganalisis sebuah model matematika penyebaran penyalahgunaan narkoba di Indonesia dengan memasukkan mekanisme rehabilitasi (rawat inap dan rawat jalan), \textit{relapse}, kematian akibat narkoba, serta laju penularan melalui interaksi sosial; model tersebut merupakan modifikasi dari yang dikembangkan oleh Mushanyu et al. dan diadaptasi untuk menggambarkan transisi antar kompartemen serta efek kapasitas rehabilitasi. Dengan menetapkan variabel proporsi individu rentan, pengguna (dengan/tanpa rehabilitasi), dan pasien rehabilitasi, dilakukan penetapan parameter, simulasi numerik, serta analisis titik kesetimbangan dan linearisasi untuk menentukan kestabilan sistem. Hasil menunjukkan adanya titik kesetimbangan bebas narkoba yang merepresentasikan kondisi ideal tanpa pengguna, namun analisis linear menunjukkan ketidakstabilan di sekitar titik tersebut; simulasi numerik juga menandakan bahwa peningkatan proporsi rekrutmen ke rehabilitasi dapat menekan laju penularan dan mengurangi prevalensi pengguna, sehingga kebijakan peningkatan akses dan rekrutmen rehabilitasi diperkirakan efektif untuk mitigasi penyebaran. Temuan ini menegaskan pentingnya program rehabilitasi yang memadai dan kebijakan intervensi sosial untuk mengendalikan dinamika penyalahgunaan narkoba.


% Ubah kata-kata berikut dengan kata kunci dari tugas akhir
\bigskip
\noindent\textbf{Kata Kunci:} model matematika, penyalahgunaan narkoba, rehabilitasi (rawat inap/rawat jalan), stabilitas, simulasi numerik, titik kesetimbangan.
