\chapter{PENDAHULUAN}
\label{chap:pendahuluan}


\section{Latar Belakang}
\label{sec:latarbelakang}
Penyalahgunaan narkoba merukan salah satu masalah sosial yang terus terjadi di Indonesia dan memberikan dampak negatif yang luas, baik untuk individu, keluarga, maupun masyarakat secara keseluruhan. Berbagai laporan menunjukkan bahwa kelompok usia produktif, terutama mereka yang berusia di atas 15 tahun, memiliki potensi yang besar untuk terpapar penyalahgunaan narkoba akibat interaksi sosial di lingkungan sekitar. Faktor pergaulan, tekanan sosial, serta kemudahan akses terhadap zat adiktif membuat penyalahgunaan narkoba sulit diberantas sepenuhnya. Kondisi ini menuntut adanya strategi penanganan yang sistematis dan berkelanjutan. 

Pemerintah Indonesia telah melakukan berbagai upaya untuk menekan angka penyalahguna narkoba, salah satunya melalui program rehabilitasi yang mencakup layanan rawat jalan maupun rawat inap. Program rehabilitasi ini bertujuan memulihkan kondisi fisik dan mental pengguna agar dapat kembali berfungsi di masyarakat tanpa ketergantungan terhadap zat berbahaya. Meski demikian, keberhasilan rehabilitasi sering kali dipengaruhi oleh berbagai faktor seperti tingkat keterlibatan pasien, lama perawatan, dukungan keluarga, dan risiko \textit{relapse}. Oleh karena itu, analisis komprehensif sangat diperlukan untuk memahami efektivitas program tersebut.

Dalam konteks inilah pendekatan pemodelan matematika menjadi sangat relevan. Pemodelan matematika memungkinkan untuk merepresentasikan dinamika penyebaran penyalahgunaan narkoba secara sistematis melalui variabel dan parameter yang saling berkaitan. Dengan membangun sebuah model yang menggambarkan alur transisi seperti individu rentan, pengguna aktif, pasien rehabilitasi, hingga kemungkinan \textit{relapse}, sehingga dapat memahami interaksi antar kompartemen tersebut. Selain itu, model mampu mensimulasikan perubahan populasi pengguna akibat intervensi tertentu, termasuk program rehabilitasi.

Melalui analisis model, terutama dengan meninjau titik kesetimbangan dan kestabilannya, dapat diketahui apakah suatu kondisi bebas narkoba secara teori mungkin tercapai dan apakah kondisi tersebut stabil terhadap gangguan kecil. Simulasi numerik juga dapat menunjukkan bagaimana perubahan parameter, seperti peningkatan proporsi individu yang direkrut ke rehabilitasi atau penurunan laju relapse, memengaruhi dinamika penyebaran narkoba secara keseluruhan. Hasil semacam ini sangat penting untuk memahami efektivitas intervensi secara kuantitatif.

Secara keseluruhan, penggunaan pemodelan matematika dalam mempelajari penyebaran penyalahgunaan narkoba memberikan manfaat besar bagi penyusunan kebijakan. Dengan adanya model yang mampu menggambarkan dinamika penyebaran secara realistis, pembuat kebijakan dapat merumuskan strategi yang lebih terarah dan efektif. Temuan dari model dapat memberikan gambaran mengenai pentingnya peningkatan akses rehabilitasi, penguatan edukasi masyarakat, serta perlunya intervensi sosial yang komprehensif untuk menekan angka penyalahgunaan narkoba di Indonesia.


\section{Rumusan Masalah}
\label{sec:permasalahan}
Adapun permasalahan yang kita angkat untuk model "Penyebaran penyalahguna narkoba di Indonesia dengan mempertimbangkan usaha rehabilitasi" antara lain:
\begin{enumerate}
    \item Bagaimana penyebaran penyalahgunaan narkoba di masyarakat dapat dimodelkan secara matematis?
    \item Bagaimana pengaruh upaya rehabilitasi, baik rawat jalan maupun rawat inap, terhadap dinamika populasi penyalahguna narkoba?
    \item Bagaimana kestabilan titik kesetimbangan pada model tersebut, khususnya kondisi bebas narkoba?
    \item Faktor apa saja yang paling berpengaruh terhadap peningkatan atau penurunan jumlah penyalahguna narkoba berdasarkan hasil simulasi?
\end{enumerate}

\section{Tujuan}
\label{sec:Tujuan}
Adapun tujuan kita memodelkan "Penyebaran penyalahguna narkoba di Indonesia dengan mempertimbangkan usaha rehabilitasi" antara lain:
\begin{enumerate}
\item Mengetahui dan menganalisis bagaimana program rehabilitasi, baik rawat jalan maupun rawat inap, memengaruhi dinamika populasi penyalahguna narkoba dalam model.
\item Mengetahui dan menganalisis bagaimana program rehabilitasi, baik rawat jalan maupun rawat inap, memengaruhi dinamika populasi penyalahguna narkoba dalam model.
\item Menentukan dan mengevaluasi kestabilan titik kesetimbangan, khususnya kondisi bebas narkoba, berdasarkan analisis dari model
\item Mengidentifikasi faktor yang paling signifikan dalam meningkatkan atau menurunkan jumlah penyalahguna narkoba melalui hasil simulasi numerik.
\end{enumerate}


\section{Batasan Masalah}
\label{sec: Batasan Masalah}
Adapun batasan masalah yang menjadi lingkup pembahasan kita untuk model "Penyebaran penyalahguna narkoba di Indonesia dengan mempertimbangkan usaha rehabilitasi" antara lain:
\begin{enumerate}
    \item Model  hanya mempertimbangkan individu berusia di atas 15 tahun
    \item Rehabilitasi dibagi menjadi dua jenis: rawat jalan dan rawat inap
    \item Tidak mempertimbangkan faktor sosial (tekanan sosial untuk mencoba narkoba)
    \item Kapasitas rehabilitasi dianggap tidak terbatas
    \item Faktor lain seperti ekonomi, pendidikan, dan penegakan hukum tidak dimasukkan dalam model
\end{enumerate}

\section{Manfaat}
\label{sec:batasanmasalah}
Pemodelan ini diharapkan dapat memberikan manfaat untuk menganalisis  penyebaran penyalahgunaan narkoba dan peran rehabilitasi di Indonesia
\begin{enumerate}
    \item Memberikan gambaran matematis mengenai penyebaran penyalahgunaan narkoba di Indonesia
    \item Membantu instansi terkait dalam memprediksi dinamika jumlah penyalahguna
    \item Mengembangkan wawasan mahasiswa mengenai penerapan pemodelan matematika dalam permasalahan sosial
\end{enumerate}

