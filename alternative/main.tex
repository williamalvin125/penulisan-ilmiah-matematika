
% Judul dokumen
\title{Buku Tugas Akhir ITS}
\author{Musk, Elon Reeve}

% Pengaturan ukuran teks dan bentuk halaman dua sisi
\documentclass[12pt,twoside]{report}

% Pengaturan ukuran halaman dan margin
\usepackage[a4paper,top=30mm,left=30mm,right=20mm,bottom=25mm]{geometry}

% Pengaturan ukuran spasi
\usepackage[singlespacing]{setspace}

% Pengaturan detail pada file PDF
\usepackage[pdfauthor={\@author},bookmarksnumbered,pdfborder={0 0 0}]{hyperref}

% Pengaturan jenis karakter
\usepackage[utf8]{inputenc}

% Pengaturan pewarnaan
\usepackage[table,xcdraw]{xcolor}

% Package lainnya
\usepackage{changepage}
\usepackage{enumitem}
\usepackage{eso-pic}
\usepackage{txfonts} % Font times
\usepackage{etoolbox}
\usepackage{graphicx}
\usepackage{lipsum}
\usepackage{longtable}
\usepackage{tabularx}
\usepackage{wrapfig}
\usepackage{float}
\usepackage{lipsum}
\usepackage{acronym}
\usepackage[T1]{fontenc}
\usepackage{titlesec}
\usepackage{ragged2e} 
\usepackage{relsize}
\usepackage{indentfirst}
\usepackage{amsmath, amssymb, amsfonts}
\usepackage{natbib} 
\usepackage{listings}
\lstset{
  language=Matlab,
  backgroundcolor=\color{gray!10},
  basicstyle=\ttfamily\footnotesize,
  keywordstyle=\color{blue}\bfseries,
  commentstyle=\color{gray},
  stringstyle=\color{green!50!black},
  showstringspaces=false,
  frame=single,
  numbers=left,
  numberstyle=\tiny,
  breaklines=true,
  tabsize=4
}

\documentclass{standalone}
\usepackage{tikz}
\usetikzlibrary{arrows.meta, positioning}

\patchcmd{\cleardoublepage}{\hbox{}}{
  \thispagestyle{empty}
  \vspace*{\fill}
  \begin{center}\textit{[Halaman ini sengaja dikosongkan]}\end{center}
  \vfill}{}{}

% Pengaturan penomoran halaman
\usepackage{fancyhdr}
\fancyhf{}
\renewcommand{\headrulewidth}{0pt}
\pagestyle{fancy}
\fancyfoot[LE,RO]{\thepage}
\patchcmd{\chapter}{plain}{fancy}{}{}
\patchcmd{\chapter}{empty}{plain}{}{}

% Command untuk bulan
\newcommand{\MONTH}{%
  \ifcase\the\month
  \or Januari% 1
  \or Februari% 2
  \or Maret% 3
  \or April% 4
  \or Mei% 5
  \or Juni% 6
  \or Juli% 7
  \or Agustus% 8
  \or September% 9
  \or Oktober% 10
  \or November% 11
  \or Desember% 12
  \fi}
\newcommand{\ENGMONTH}{%
  \ifcase\the\month
  \or January% 1
  \or February% 2
  \or March% 3
  \or April% 4
  \or May% 5
  \or June% 6
  \or July% 7
  \or August% 8
  \or September% 9
  \or October% 10
  \or November% 11
  \or December% 12
  \fi}

% Pengaturan format judul bab
\usepackage{titlesec}
\titleformat{\chapter}[display]{\bfseries\Large}{BAB \centering\Roman{chapter}}{0ex}{\vspace{0ex}\centering}
\titleformat{\section}{\bfseries\large}{\MakeUppercase{\thesection}}{1ex}{\vspace{1ex}}
\titleformat{\subsection}{\bfseries\large}{\MakeUppercase{\thesubsection}}{1ex}{}
\titleformat{\subsubsection}{\bfseries\large}{\MakeUppercase{\thesubsubsection}}{1ex}{}
\titlespacing{\chapter}{0ex}{0ex}{4ex}
\titlespacing{\section}{0ex}{1ex}{0ex}
\titlespacing{\subsection}{0ex}{0.5ex}{0ex}
\titlespacing{\subsubsection}{0ex}{0.5ex}{0ex}

% Tambahkan format tanda hubung yang benar di sini
\hyphenation{
  me-ne-rap-kan
}

% Isi keseluruhan dokumen
\begin{document}

% Sampul luar Bahasa Indonesia
\newcommand\covercontents{konten-id.tex}
\input{sampul-luar.tex}

% Atur ulang penomoran halaman
\setcounter{page}{1}

% Sampul dalam Bahasa Indonesia
\renewcommand\covercontents{konten-id.tex}
\input{sampul-luar-tipis.tex}
\clearpage
\cleardoublepage

% Pengaturan ukuran indentasi paragraf
\setlength{\parindent}{2em}

% Pengaturan ukuran spasi paragraf
\setlength{\parskip}{1ex}

% Nomor halaman pembuka dimulai dari sini
\pagenumbering{roman}

% Abstrak Bahasa Indonesia
\begin{center}
  \large\textbf{ABSTRAK}
\end{center}

\addcontentsline{toc}{chapter}{ABSTRAK}

\vspace{2ex}

\begingroup
% Menghilangkan padding
\setlength{\tabcolsep}{0pt}


% Ubah paragraf berikut dengan abstrak dari tugas akhir
Topik ini menyusun dan menganalisis sebuah model matematika penyebaran penyalahgunaan narkoba di Indonesia dengan memasukkan mekanisme rehabilitasi (rawat inap dan rawat jalan), \textit{relapse}, kematian akibat narkoba, serta laju penularan melalui interaksi sosial; model tersebut merupakan modifikasi dari yang dikembangkan oleh Mushanyu et al. dan diadaptasi untuk menggambarkan transisi antar kompartemen serta efek kapasitas rehabilitasi. Dengan menetapkan variabel proporsi individu rentan, pengguna (dengan/tanpa rehabilitasi), dan pasien rehabilitasi, dilakukan penetapan parameter, simulasi numerik, serta analisis titik kesetimbangan dan linearisasi untuk menentukan kestabilan sistem. Hasil menunjukkan adanya titik kesetimbangan bebas narkoba yang merepresentasikan kondisi ideal tanpa pengguna, namun analisis linear menunjukkan ketidakstabilan di sekitar titik tersebut; simulasi numerik juga menandakan bahwa peningkatan proporsi rekrutmen ke rehabilitasi dapat menekan laju penularan dan mengurangi prevalensi pengguna, sehingga kebijakan peningkatan akses dan rekrutmen rehabilitasi diperkirakan efektif untuk mitigasi penyebaran. Temuan ini menegaskan pentingnya program rehabilitasi yang memadai dan kebijakan intervensi sosial untuk mengendalikan dinamika penyalahgunaan narkoba.


% Ubah kata-kata berikut dengan kata kunci dari tugas akhir
\bigskip
\noindent\textbf{Kata Kunci:} model matematika, penyalahgunaan narkoba, rehabilitasi (rawat inap/rawat jalan), stabilitas, simulasi numerik, titik kesetimbangan.

\cleardoublepage

% Abstrak Bahasa Inggris
\begin{center}
  \large\textbf{ABSTRACT}
\end{center}

\addcontentsline{toc}{chapter}{ABSTRACT}

\vspace{2ex}

\begingroup
% Menghilangkan padding
\setlength{\tabcolsep}{0pt}

% Ubah paragraf berikut dengan abstrak dari tugas akhir dalam Bahasa Inggris
This study formulates and analyzes a mathematical model for the spread of drug abuse in Indonesia by incorporating rehabilitation mechanisms (inpatient and outpatient), relapse, drug-related mortality, and transmission rates through social interactions; the model is a modification of the framework developed by Mushanyu et al. and is adapted to represent transitions between compartments as well as the effects of rehabilitation capacity. By defining variables as the proportions of susceptible individuals, users (with/without rehabilitation), and rehabilitation patients, the study sets model parameters, carries out numerical simulations, and performs equilibrium-point analysis and linearization to determine system stability. Results indicate the existence of a drug-free equilibrium that represents the ideal state without users; however, linear analysis shows instability around this point. Numerical simulations further suggest that increasing the recruitment proportion into rehabilitation can suppress transmission rates and reduce user prevalence, meaning that policies to expand access to and recruitment for rehabilitation are likely effective for mitigation. These findings underscore the importance of adequate rehabilitation programs and social-intervention policies to control the dynamics of drug abuse.

% Ubah kata-kata berikut dengan kata kunci dari tugas akhir dalam Bahasa Inggris
\bigskip
\noindent\textbf{Keywords:} mathematical model, drug abuse, rehabilitation (inpatient/outpatient), stability, numerical simulation, equilibrium point.

\cleardoublepage

%Pendahuluan
\chapter{PENDAHULUAN}
\label{chap:pendahuluan}


\section{Latar Belakang}
\label{sec:latarbelakang}
Penyalahgunaan narkoba merukan salah satu masalah sosial yang terus terjadi di Indonesia dan memberikan dampak negatif yang luas, baik untuk individu, keluarga, maupun masyarakat secara keseluruhan. Berbagai laporan menunjukkan bahwa kelompok usia produktif, terutama mereka yang berusia di atas 15 tahun, memiliki potensi yang besar untuk terpapar penyalahgunaan narkoba akibat interaksi sosial di lingkungan sekitar. Faktor pergaulan, tekanan sosial, serta kemudahan akses terhadap zat adiktif membuat penyalahgunaan narkoba sulit diberantas sepenuhnya. Kondisi ini menuntut adanya strategi penanganan yang sistematis dan berkelanjutan. 

Pemerintah Indonesia telah melakukan berbagai upaya untuk menekan angka penyalahguna narkoba, salah satunya melalui program rehabilitasi yang mencakup layanan rawat jalan maupun rawat inap. Program rehabilitasi ini bertujuan memulihkan kondisi fisik dan mental pengguna agar dapat kembali berfungsi di masyarakat tanpa ketergantungan terhadap zat berbahaya. Meski demikian, keberhasilan rehabilitasi sering kali dipengaruhi oleh berbagai faktor seperti tingkat keterlibatan pasien, lama perawatan, dukungan keluarga, dan risiko \textit{relapse}. Oleh karena itu, analisis komprehensif sangat diperlukan untuk memahami efektivitas program tersebut.

Dalam konteks inilah pendekatan pemodelan matematika menjadi sangat relevan. Pemodelan matematika memungkinkan untuk merepresentasikan dinamika penyebaran penyalahgunaan narkoba secara sistematis melalui variabel dan parameter yang saling berkaitan. Dengan membangun sebuah model yang menggambarkan alur transisi seperti individu rentan, pengguna aktif, pasien rehabilitasi, hingga kemungkinan \textit{relapse}, sehingga dapat memahami interaksi antar kompartemen tersebut. Selain itu, model mampu mensimulasikan perubahan populasi pengguna akibat intervensi tertentu, termasuk program rehabilitasi.

Melalui analisis model, terutama dengan meninjau titik kesetimbangan dan kestabilannya, dapat diketahui apakah suatu kondisi bebas narkoba secara teori mungkin tercapai dan apakah kondisi tersebut stabil terhadap gangguan kecil. Simulasi numerik juga dapat menunjukkan bagaimana perubahan parameter, seperti peningkatan proporsi individu yang direkrut ke rehabilitasi atau penurunan laju relapse, memengaruhi dinamika penyebaran narkoba secara keseluruhan. Hasil semacam ini sangat penting untuk memahami efektivitas intervensi secara kuantitatif.

Secara keseluruhan, penggunaan pemodelan matematika dalam mempelajari penyebaran penyalahgunaan narkoba memberikan manfaat besar bagi penyusunan kebijakan. Dengan adanya model yang mampu menggambarkan dinamika penyebaran secara realistis, pembuat kebijakan dapat merumuskan strategi yang lebih terarah dan efektif. Temuan dari model dapat memberikan gambaran mengenai pentingnya peningkatan akses rehabilitasi, penguatan edukasi masyarakat, serta perlunya intervensi sosial yang komprehensif untuk menekan angka penyalahgunaan narkoba di Indonesia.


\section{Rumusan Masalah}
\label{sec:permasalahan}
Adapun permasalahan yang kita angkat untuk model "Penyebaran penyalahguna narkoba di Indonesia dengan mempertimbangkan usaha rehabilitasi" antara lain:
\begin{enumerate}
    \item Bagaimana penyebaran penyalahgunaan narkoba di masyarakat dapat dimodelkan secara matematis?
    \item Bagaimana pengaruh upaya rehabilitasi, baik rawat jalan maupun rawat inap, terhadap dinamika populasi penyalahguna narkoba?
    \item Bagaimana kestabilan titik kesetimbangan pada model tersebut, khususnya kondisi bebas narkoba?
    \item Faktor apa saja yang paling berpengaruh terhadap peningkatan atau penurunan jumlah penyalahguna narkoba berdasarkan hasil simulasi?
\end{enumerate}

\section{Tujuan}
\label{sec:Tujuan}
Adapun tujuan kita memodelkan "Penyebaran penyalahguna narkoba di Indonesia dengan mempertimbangkan usaha rehabilitasi" antara lain:
\begin{enumerate}
\item Mengetahui dan menganalisis bagaimana program rehabilitasi, baik rawat jalan maupun rawat inap, memengaruhi dinamika populasi penyalahguna narkoba dalam model.
\item Mengetahui dan menganalisis bagaimana program rehabilitasi, baik rawat jalan maupun rawat inap, memengaruhi dinamika populasi penyalahguna narkoba dalam model.
\item Menentukan dan mengevaluasi kestabilan titik kesetimbangan, khususnya kondisi bebas narkoba, berdasarkan analisis dari model
\item Mengidentifikasi faktor yang paling signifikan dalam meningkatkan atau menurunkan jumlah penyalahguna narkoba melalui hasil simulasi numerik.
\end{enumerate}


\section{Batasan Masalah}
\label{sec: Batasan Masalah}
Adapun batasan masalah yang menjadi lingkup pembahasan kita untuk model "Penyebaran penyalahguna narkoba di Indonesia dengan mempertimbangkan usaha rehabilitasi" antara lain:
\begin{enumerate}
    \item Model  hanya mempertimbangkan individu berusia di atas 15 tahun
    \item Rehabilitasi dibagi menjadi dua jenis: rawat jalan dan rawat inap
    \item Tidak mempertimbangkan faktor sosial (tekanan sosial untuk mencoba narkoba)
    \item Kapasitas rehabilitasi dianggap tidak terbatas
    \item Faktor lain seperti ekonomi, pendidikan, dan penegakan hukum tidak dimasukkan dalam model
\end{enumerate}

\section{Manfaat}
\label{sec:batasanmasalah}
Pemodelan ini diharapkan dapat memberikan manfaat untuk menganalisis  penyebaran penyalahgunaan narkoba dan peran rehabilitasi di Indonesia
\begin{enumerate}
    \item Memberikan gambaran matematis mengenai penyebaran penyalahgunaan narkoba di Indonesia
    \item Membantu instansi terkait dalam memprediksi dinamika jumlah penyalahguna
    \item Mengembangkan wawasan mahasiswa mengenai penerapan pemodelan matematika dalam permasalahan sosial
\end{enumerate}



%Tinjauan Pustaka
\chapter{TINJAUAN PUSTAKA}
\label{chap:tinjauanpustaka}
Penelitian mengenai penyebaran penyalahgunaan narkoba telah banyak dikasi melalui pendekatan pemodelan matematika untuk memahami dinamika interaksi antar kelompok dalam polukasi. Salah satu penelitian yang dilakukan oleh Mushanyu(2018) yang mengembangkan model penyebaran penyalahgunaan narkoba dengan mempertimbangkan efek imitasi sosial serta keterbatasan kapasitas rehabilitasi. Model tersebut menunjukkan bahwa peningkatan kapasitas rehabilitasi dapat mengurangi jumlah penyalahguna narkoba dalam jangka panjang

\section{Penelitian Terdahulu}
Beberapa studi kunci yang menjadi acuan dalam presentasi ini adalah karya-karya Mushanyu dan rekan (2018) yang mengeksplorasi penyebaran penyalahgunaan narkoba dengan mempertimbangkan faktor imitasi sosial dan keterbatasan kapasitas rehabilitasi. Ringkasan rujukan utama adalah sebagai berikut:

\begin{itemize}[leftmargin=*,itemsep=6pt]
    \item Mushanyu (2018): Mengembangkan model \textit{epidemic-like} yang memasukkan mekanisme imitasi sosial dan membahas bagaimana kapasitas rehabilitasi mempengaruhi efektivitas intervensi. Model ini menjadi acuan utama untuk modifikasi pada presentasi kelompok. 
    \item Mushanyu \& Nyabadza (2018): Memaparkan model berstruktur risiko (\textit{risk-structured model}) yang membagi populasi berdasarkan tingkat risiko, dan menjelaskan pola transisi antar-kompartemen yang relevan untuk memahami dinamika penyalahgunaan.
    \item Sumber data dan laporan kebijakan nasional: Dokumen BNN (Infografis P4GN), data demografi BPS (jumlah penduduk menurut kelompok umur), serta laporan rumah sakit dan statistik mortalitas yang digunakan sebagai sumber nilai parameter atau validasi asumsi.
\end{itemize}

\section{Temuan Penelitian yang Relevan}
Berdasarkan kajian literatur utama (khususnya karya Mushanyu dan rekan), ada beberapa temuan penting yang relevan untuk pengembangan model:
\begin{enumerate}[leftmargin=*,itemsep=4pt]
    \item Interaksi sosial dan imitasi dapat mempercepat penyebaran perilaku penyalahgunaan narkoba di antara individu rentan.
    \item Keterbatasan kapasitas rehabilitasi (misalnya jumlah tempat perawatan rawat inap) mengurangi efektivitas program rehabilitasi bila permintaan melebihi kapasitas.
    \item Pembagian populasi menurut tingkat risiko (struktur risiko) mengubah pola transisi antara status rentan, pengguna, dan pasien rehabilitasi.
\end{enumerate}

Temuan-temuan tersebut menjadi landasan teori mengapa model kompartemental yang memuat kompartemen rehabilitasi dan \textit{relapse} relevan untuk dianalisis pada konteks Indonesia.

\section{Modifikasi Model}
Dibandingkan model awal, dilakukan beberapa modifikasi penting:
\begin{enumerate}[leftmargin=*,itemsep=4pt]
    \item Pada beberapa varian model, efek imitasi sosial diabaikan untuk menyederhanakan analisis sehingga transmisi difokuskan pada kontak langsung antara individu rentan dan pengguna.
    \item Terdapat variasi model yang mengabaikan keterbatasan kapasitas rehabilitasi, sementara variasi lain memasukkan kembali pembatas kapasitas sebagai fungsi pembatas rekrutmen.
    \item Ditambahkan laju kematian akibat narkoba sebagai komponen mortalitas spesifik pengguna, guna mempertajam representasi dinamika populasi.
\end{enumerate}

Tujuan modifikasi ini adalah menyeimbangkan kompleksitas matematis dengan ketersediaan data dan tujuan analisis (mis. memisahkan pengaruh rawat jalan vs rawat inap, mengeksplorasi peran \textit{relapse}, dan lain-lain).

\section{Rekomendasi Pengembangan}
\begin{itemize}[leftmargin=*,itemsep=4pt]
    \item Model-model terdahulu (terutama karya Mushanyu et al.) memberikan kerangka kuat untuk memasukkan efek imitasi sosial dan pembatas kapasitas rehabilitasi.
    \item Penerapan model pada konteks Indonesia membutuhkan penyesuaian asumsi dan estimasi parameter berbasis data lokal (BNN, BPS, data rumah sakit).
    \item Rekomendasi pengembangan meliputi: (i) eksplorasi skenario dengan/ tanpa efek imitasi dan kapasitas; (ii) estimasi parameter dari data lapangan untuk validasi; (iii) analisis sensitivitas kuantitatif untuk mengidentifikasi parameter paling krusial.
\end{itemize}

\section{Stabilitas Dilihat dari Nilai Eigen}
Analisis stabilitas merupakan salah satu konsep fundamental dalam kajian sistem dinamis, baik kontinu maupun diskrit. Secara umum, suatu sistem dikatakan stabil apabila solusi yang berada di sekitar titik kesetimbangan tetap berada di dekat titik tersebut atau kembali ke titik tersebut seiring berjalannya waktu. Dalam konteks sistem dinamis nonlinier, dua pendekatan yang sering digunakan adalah analisis nilai eigen dari matriks Jacobian dan metode linearisasi di sekitar titik kesetimbangan.

Untuk sistem dinamis linear
\begin{equation}
    \dot{x}(t) = A x(t),
\end{equation}
perilaku solusi sepenuhnya ditentukan oleh nilai eigen matriks $A$. Jika $\lambda_1, \lambda_2, \dots, \lambda_n$ adalah nilai eigen dari $A$, maka solusi umum dapat dinyatakan sebagai
\begin{equation}
    x(t) = c_1 e^{\lambda_1 t} v_1 + c_2 e^{\lambda_2 t} v_2 + \dots + c_n e^{\lambda_n t} v_n,
\end{equation}
dengan $v_i$ adalah vektor eigen terkait.

Kriteria stabilitasnya adalah sebagai berikut:
\begin{enumerate}
    \item Jika seluruh nilai eigen memenuhi $\Re(\lambda_i) < 0$, maka sistem \textbf{stabil asimtotik}.
    \item Jika terdapat nilai eigen dengan $\Re(\lambda_i) > 0$, maka sistem \textbf{tidak stabil}.
    \item Jika seluruh nilai eigen memenuhi $\Re(\lambda_i) \le 0$ dan sedikitnya satu dengan $\Re(\lambda_i) = 0$, maka kestabilan \textbf{bergantung pada orde Jordan} (disebut stabilitas netral atau kritis).
\end{enumerate}

Analisis nilai eigen memberikan cara yang sederhana dan efektif untuk menentukan kestabilan sistem linear. Oleh karena itu, metode ini juga digunakan untuk analisis awal pada sistem nonlinier melalui pendekatan linearisasi.

\section{Linearisasi Sistem Dinamis Nonlinier}
Pertimbangkan sistem nonlinier umum:
\begin{equation}
    \dot{x}(t) = f(x), \qquad x \in \mathbb{R}^n,
\end{equation}
dengan titik kesetimbangan $x^\ast$ yang memenuhi $f(x^\ast)=0$. Pendekatan linearisasi dilakukan dengan memperluas fungsi $f(x)$ menggunakan deret Taylor di sekitar titik keseimbangan:
\begin{equation}
    f(x) \approx f(x^\ast) + J_f(x^\ast)(x - x^\ast),
\end{equation}
di mana $J_f(x^\ast)$ adalah matriks Jacobian yang didefinisikan sebagai
\begin{equation}
    J_f(x^\ast) = 
    \begin{bmatrix}
        \frac{\partial f_1}{\partial x_1} & \cdots & \frac{\partial f_1}{\partial x_n} \\
        \vdots & \ddots & \vdots \\
        \frac{\partial f_n}{\partial x_1} & \cdots & \frac{\partial f_n}{\partial x_n}
    \end{bmatrix}_{x = x^\ast}.
\end{equation}

Karena $f(x^\ast)=0$, sistem dinamis dapat didekati oleh sistem linear:
\begin{equation}
    \dot{y}(t) = J_f(x^\ast) y(t),
\end{equation}
dengan $y(t) = x(t) - x^\ast$.

\section{Stabilitas Sistem Nonlinier Berdasarkan Linearisasi}
Untuk sistem nonlinier, kestabilan titik kesetimbangan dapat diprediksi berdasarkan sifat nilai eigen dari matriks Jacobian. Kriteria utamanya adalah:
\begin{enumerate}
    \item Jika seluruh nilai eigen dari $J_f(x^\ast)$ memenuhi $\Re(\lambda_i) < 0$, maka titik kesetimbangan \textbf{stabil asimtotik}.
    \item Jika terdapat nilai eigen dengan $\Re(\lambda_i) > 0$, maka titik kesetimbangan \textbf{tidak stabil}.
    \item Jika ada nilai eigen dengan $\Re(\lambda_i) = 0$, maka linearisasi \textbf{tidak cukup untuk menentukan stabilitas}; analisis lanjutan diperlukan seperti metode Lyapunov atau higher-order analysis.
\end{enumerate}

Pendekatan linearisasi ini sangat berguna karena dapat menyederhanakan analisis sistem nonlinier yang kompleks menjadi sistem linear yang lebih mudah dipelajari. Metode ini banyak digunakan dalam studi epidemiologi, dinamika populasi, dinamika penyalahgunaan narkoba, dan berbagai bidang rekayasa sistem.

\newpage

\chapter{METODE PENELITIAN}
\label{chap:metodepenelitian}

\section{Langkah-langkah Pemodelan}
\label{sec:langkah-langkahpemodelan}

\subsection{Objek}
\label{subsec:objek}
Adapun objek yang kita modelkan yaitu "Penyebaran penyalahguna narkoba di Indonesia dengan mempertimbangkan usaha rehabilitasi".

\subsection{Fenomena}
\label{subsec:fenomena}
Fenomena yang kita gunakan dalam pemodelan "Penyebaran penyalahguna narkoba di Indonesia dengan mempertimbangkan usaha rehabilitasi" yaitu:
 \begin{enumerate}

    \item Setiap individu di segala usia berpotensi terpapar penyalahgunaan narkoba
    \item Setiap individu rentan dapat terpapar penyalahgunaan narkoba melalui interaksi dengan penyalahguna narkoba
    \item Banyaknya penyalahguna narkoba di lingkungan sosial dapat meningkatkan laju kenaikan jumlah penyalahguna narkoba
    \item Penyalahguna narkoba diarahkan untuk menjalani proses rehabilitasi
    \item Rehabilitasi dilakukan dalam bentuk rawat jalan dan rawat inap
    \item Penggunaan narkoba secara berlebihan dapat menyebabkan kematian
    \item Setiap individu dapat sembuh dan berhenti menggunakan narkoba secara permanen
    \item Pasien rehabilitasi berpotensi berhenti mengikuti proses rehabilitasi
    \item Pasien rehabilitasi dapat berpindah dari rehabilitasi rawat jalan ke rawat inap, dan sebaliknya
    \item Laju rekrutmen pasien rehabilitasi dipengaruhi oleh kapasitas rehabilitasi
    
\end{enumerate}

\subsection{Hukum}
\label{subsec:hukum}
Pemodelan ini menggunakan hukum interaksi.
\subsection{Asumsi}
Asumsi yang kita gunakan yaitu:
\begin{enumerate}
\item Setiap individu yang berusia di atas 15 tahun berpotensi terpapar penyalahgunaan narkoba
\item Setiap individu rentan dapat terpapar penyalahgunaan narkoba melalui interaksi dengan penyalahguna narkoba
\item Banyaknya penyalahguna narkoba di lingkungan sosial tidak meningkatkan laju kenaikan jumlah penyalahguna narkoba
\item Penyalahguna narkoba diarahkan untuk menjalani proses rehabilitasi
\item Rehabilitasi dilakukan dalam bentuk rawat jalan dan rawat inap
\item Penggunaan narkoba secara berlebihan dapat menyebabkan kematian
\item Setiap individu dapat sembuh dan berhenti menggunakan narkoba secara permanen setelah melakukan rehabilitasi.
\item Pasien rehabilitasi berpotensi berhenti mengikuti proses rehabilitasi
\item Pasien rehabilitasi dapat berpindah dari rehabilitasi rawat jalan ke rawat inap, dan sebaliknya
\item Laju rekrutmen pasien rehabilitasi tidak dipengaruhi oleh kapasitas rehabilitasi

\end{enumerate}

\subsection{Variabel}
\label{subsec:variabel}
Variabel yang ada pada pemodelan ini antara lain:
\renewcommand{\arraystretch}{1.125}
\begingroup
\allowdisplaybreaks
\begin{table}[H]
\begin{center}
%% increase table row spacing, adjust to taste
\renewcommand{\arraystretch}{1.125}
\caption{Variabel}
\centering
\begin{tabular}{|c|c|}
\hline
\ Variabel \ & Keterangan \\
\hline
$S(t)$ & $\begin{matrix}\text{Proporsi individu yang  } \\
\text{ rentan terpapar penyalahgunaan } \\
\text{narkoba pada tahun ke t}\end{matrix}$\\
\hline
$U(t)$ & $\begin{matrix}\text{Proporsi individu penyalahguna } \\
\text{narkoba yang tidak menjalani} \\
\text{rehabilitasi pada tahun ke t}\end{matrix}$\\
\hline
$R_{op}(t)$ & $\begin{matrix}\text{Proporsi individu penyalahguna } \\
\text{ \ narkoba yang menjalani rehabilitasi \ }\\
\text{ rawat jalan pada tahun ke t}\end{matrix}$ \\
\hline
$R_{ip}(t)$ & $\begin{matrix}\text{Proporsi individu penyalahguna } \\
\text{narkoba yang menjalani rehabilitasi}\\
\text{ rawat inap pada tahun ke t}\end{matrix}$ \\
\hline
\end{tabular}
\end{center}
\end{table}
\endgroup


\subsection{Parameter}

Parameter yang ada pada pemodelan ini antara lain:

\begingroup
\allowdisplaybreaks
\begin{table}[H]
\begin{center}
%% increase table row spacing, adjust to taste
\renewcommand{\arraystretch}{1.125}
\caption{Parameter}
\centering
\begin{tabular}{|c|p{10cm}|}
\hline
\ Parameter \  & Keterangan \\
\hline
$\Lambda$ & $\begin{matrix}\text{\ Jumlah individu yang masuk ke sistem}\end{matrix}$ \\
\hline
$\beta_1$ & $\begin{matrix}\text{\ Laju penularan melalui interaksi S-U}\end{matrix}$ \\
\hline
$\beta_2$ & $\begin{matrix}\text{\ Laju penularan melalui interaksi S-Rop}\end{matrix}$ \\
\hline
$\mu$ & $\begin{matrix}\text{\ Laju kematian alami}\end{matrix}$ \\
\hline
$\rho_1$ & $\begin{matrix}\text{\ Laju relapse (kambuh) dari Rop}\end{matrix}$ \\
\hline
$\rho_2$ & $\begin{matrix}\text{\ Laju relapse (kambuh) dari Rip}\end{matrix}$ \\
\hline
$P$ & $\begin{matrix}\text{\ Proporsi pengguna yang masuk rawat inap}\end{matrix}$ \\
\hline
$\alpha$ & $\begin{matrix}\text{\ Laju rekrutmen rehabilitasi dari U}\end{matrix}$ \\
\hline
$\gamma_1$& \ Laju transisi Rop ke Rip \\
\hline

$\gamma_2$ & \ Laju transisi Rip ke Rop \\
\hline
$\delta_1$ & $\begin{matrix}\text{\ Laju pemulihan dari Rop}\end{matrix}$ \\
\hline
$\delta_2$ & $\begin{matrix}\text{\ Laju pemulihan dari Rip}\end{matrix}$ \\
\hline
$\theta$ & $\begin{matrix}\text{\ Laju kematian U akibat narkoba}\end{matrix}$ \\
\hline


\end{tabular}
\end{center}
\end{table}
\endgroup
\subsection{Diagram Kompartemen}
\label{subsec:diagramkompartemen}
\begin{figure}[H]
    \centering
    \includegraphics[width=0.9\linewidth]{Untitled_Diagram.drawio__1_-removebg-preview (1).png}
    \caption{Diagram Model}
    \label{fig:enter-label}
\end{figure}


\subsection{Model}

\begin{align*}
\dfrac{dS}{dt} &= \Lambda - \beta_1 SU - \beta_2 S R_{op} - \mu S, \\
\dfrac{dU}{dt} &= \beta_1 SU + \beta_2 S R_{op} + \rho_1 R_{op} + \rho_2 R_{ip} - (\mu + \alpha + \theta)U, \\
\dfrac{dR_{op}}{dt} &= \gamma_2 R_{ip} - (\mu + \gamma_1 + \rho_1 + \delta_1)R_{op} + (1 - p)\alpha U, \\
\dfrac{dR_{ip}}{dt} &= \gamma_1 R_{op} - (\mu + \gamma_2 + \rho_2 + \delta_2)R_{ip} + p\alpha U.
\end{align*}


\subsection{ Nilai Parameter}

Nilai parameter yang ada pada pemodelan ini antara lain:

\begingroup
\allowdisplaybreaks
\begin{table}[H]
\begin{center}
%% increase table row spacing, adjust to taste
\renewcommand{\arraystretch}{1.125}
\caption{Tabel Parameter}
\centering
\begin{tabular}{|c|c|}
\hline
\ Parameter \  & Nilai \\
\hline
$\Lambda$ & $\begin{matrix}\text{\ 0.0209}\end{matrix}$ \\
\hline
$\beta_1$ & $\begin{matrix}\text{\ 0.0411}\end{matrix}$ \\
\hline
$\beta_2$ & $\begin{matrix}\text{\ 0.063}\end{matrix}$ \\
\hline
$\mu$ & $\begin{matrix}\text{\ 0.00749}\end{matrix}$ \\
\hline
$\rho_1$ & $\begin{matrix}\text{\ 0.0382}\end{matrix}$ \\
\hline
$\rho_2$ & $\begin{matrix}\text{\ 0.002}\end{matrix}$ \\
\hline
$P$ & $\begin{matrix}\text{\ 0.352}\end{matrix}$ \\
\hline
$\alpha$ & $\begin{matrix}\text{\ 0.02827}\end{matrix}$ \\
\hline
$\gamma_1$& \ 0.02961 \\
\hline

$\gamma_2$ & \ 0.003 \\
\hline
$\delta_1$ & $\begin{matrix}\text{\ 0.01}\end{matrix}$ \\
\hline
$\delta_2$ & $\begin{matrix}\text{\ 0.3142}\end{matrix}$ \\
\hline
$\theta$ & $\begin{matrix}\text{\ 0.005}\end{matrix}$ \\
\hline


\end{tabular}
\end{center}
\end{table}
\endgroup


\subsection{Simulasi Model}
tolong yang ada file simulasinya bantu masukin ya


\subsection{Titik Kesetimbangan Bebas Narkoba}
\begin{align*}
\frac{dS}{dt} &= \frac{dU}{dt} = \frac{dR_{op}}{dt} = \frac{dR_{ip}}{dt} = U = R_{op} = R_{ip} = 0
\end{align*}

% Garis panah menggunakan downarrow
\begin{center}
$\downarrow$
\end{center}

\begin{align*}
S &= \frac{\Lambda}{\mu} = 2.7903
\end{align*}

% Garis panah menggunakan downarrow
\begin{center}
$\downarrow$
\end{center}

\begin{align*}
(S, U, R_{op}, R_{sp}) &= (2.7903, 0, 0, 0)
\end{align*}

\subsection{Linearisasi Model}
\begin{equation*}
\dot{Y} = A^* Y
\end{equation*}

\begin{equation*}
\begin{pmatrix}
\dfrac{dS}{dt} \\[8pt]
\dfrac{dU}{dt} \\[8pt]
\dfrac{dR_{op}}{dt} \\[8pt]
\dfrac{dR_{tp}}{dt}
\end{pmatrix}
=
\begin{pmatrix}
-0.0075 & -0.1231 & -0.1758 & 0 \\
0 & 0.0823 & 0.2140 & 0.0020 \\
0 & 0.0183 & -0.0853 & 0.0030 \\
0 & 0.0099 & 0.0296 & -0.3267
\end{pmatrix}
\begin{pmatrix}
S \\[4pt]
U \\[4pt]
R_{op} \\[4pt]
R_{tp}
\end{pmatrix}
\end{equation*}

\subsection{Analisis Kestabilan Model Linear}
Analisis kestabilan model linear berdasarkan nilai eigen matriks \( A^* \) sebagai berikut:

\begin{equation*}
(\lambda_1, \lambda_2, \lambda_3, \lambda_4) = (-0.0075, 0.1032, -0.1059, -0.3270)
\end{equation*}

Karena terdapat nilai eigen positif (\( \lambda_2 = 0.1032 > 0 \)), maka:

\textbf{Model linear tidak stabil}
\newpage

% \chapter{Daftar Pustaka}
% \label{references}

\nocite{*} % Menampilkan semua referensi dari file .bib
\renewcommand{\bibname}{Daftar Pustaka}
\bibliographystyle{apalike}
\bibliography{daftar_pustaka}
\addcontentsline{toc}{chapter}{Daftar Pustaka}

\end{document}
