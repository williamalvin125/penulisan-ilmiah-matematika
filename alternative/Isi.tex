\chapter{METODE PENELITIAN}
\label{chap:metodepenelitian}

\section{Langkah-langkah Pemodelan}
\label{sec:langkah-langkahpemodelan}

\subsection{Objek}
\label{subsec:objek}
Adapun objek yang kita modelkan yaitu "Penyebaran penyalahguna narkoba di Indonesia dengan mempertimbangkan usaha rehabilitasi".

\subsection{Fenomena}
\label{subsec:fenomena}
Fenomena yang kita gunakan dalam pemodelan "Penyebaran penyalahguna narkoba di Indonesia dengan mempertimbangkan usaha rehabilitasi" yaitu:
 \begin{enumerate}

    \item Setiap individu di segala usia berpotensi terpapar penyalahgunaan narkoba
    \item Setiap individu rentan dapat terpapar penyalahgunaan narkoba melalui interaksi dengan penyalahguna narkoba
    \item Banyaknya penyalahguna narkoba di lingkungan sosial dapat meningkatkan laju kenaikan jumlah penyalahguna narkoba
    \item Penyalahguna narkoba diarahkan untuk menjalani proses rehabilitasi
    \item Rehabilitasi dilakukan dalam bentuk rawat jalan dan rawat inap
    \item Penggunaan narkoba secara berlebihan dapat menyebabkan kematian
    \item Setiap individu dapat sembuh dan berhenti menggunakan narkoba secara permanen
    \item Pasien rehabilitasi berpotensi berhenti mengikuti proses rehabilitasi
    \item Pasien rehabilitasi dapat berpindah dari rehabilitasi rawat jalan ke rawat inap, dan sebaliknya
    \item Laju rekrutmen pasien rehabilitasi dipengaruhi oleh kapasitas rehabilitasi
    
\end{enumerate}

\subsection{Hukum}
\label{subsec:hukum}
Pemodelan ini menggunakan hukum interaksi.
\subsection{Asumsi}
Asumsi yang kita gunakan yaitu:
\begin{enumerate}
\item Setiap individu yang berusia di atas 15 tahun berpotensi terpapar penyalahgunaan narkoba
\item Setiap individu rentan dapat terpapar penyalahgunaan narkoba melalui interaksi dengan penyalahguna narkoba
\item Banyaknya penyalahguna narkoba di lingkungan sosial tidak meningkatkan laju kenaikan jumlah penyalahguna narkoba
\item Penyalahguna narkoba diarahkan untuk menjalani proses rehabilitasi
\item Rehabilitasi dilakukan dalam bentuk rawat jalan dan rawat inap
\item Penggunaan narkoba secara berlebihan dapat menyebabkan kematian
\item Setiap individu dapat sembuh dan berhenti menggunakan narkoba secara permanen setelah melakukan rehabilitasi.
\item Pasien rehabilitasi berpotensi berhenti mengikuti proses rehabilitasi
\item Pasien rehabilitasi dapat berpindah dari rehabilitasi rawat jalan ke rawat inap, dan sebaliknya
\item Laju rekrutmen pasien rehabilitasi tidak dipengaruhi oleh kapasitas rehabilitasi

\end{enumerate}

\subsection{Variabel}
\label{subsec:variabel}
Variabel yang ada pada pemodelan ini antara lain:
\renewcommand{\arraystretch}{1.125}
\begingroup
\allowdisplaybreaks
\begin{table}[H]
\begin{center}
%% increase table row spacing, adjust to taste
\renewcommand{\arraystretch}{1.125}
\caption{Variabel}
\centering
\begin{tabular}{|c|c|}
\hline
\ Variabel \ & Keterangan \\
\hline
$S(t)$ & $\begin{matrix}\text{Proporsi individu yang  } \\
\text{ rentan terpapar penyalahgunaan } \\
\text{narkoba pada tahun ke t}\end{matrix}$\\
\hline
$U(t)$ & $\begin{matrix}\text{Proporsi individu penyalahguna } \\
\text{narkoba yang tidak menjalani} \\
\text{rehabilitasi pada tahun ke t}\end{matrix}$\\
\hline
$R_{op}(t)$ & $\begin{matrix}\text{Proporsi individu penyalahguna } \\
\text{ \ narkoba yang menjalani rehabilitasi \ }\\
\text{ rawat jalan pada tahun ke t}\end{matrix}$ \\
\hline
$R_{ip}(t)$ & $\begin{matrix}\text{Proporsi individu penyalahguna } \\
\text{narkoba yang menjalani rehabilitasi}\\
\text{ rawat inap pada tahun ke t}\end{matrix}$ \\
\hline
\end{tabular}
\end{center}
\end{table}
\endgroup


\subsection{Parameter}

Parameter yang ada pada pemodelan ini antara lain:

\begingroup
\allowdisplaybreaks
\begin{table}[H]
\begin{center}
%% increase table row spacing, adjust to taste
\renewcommand{\arraystretch}{1.125}
\caption{Parameter}
\centering
\begin{tabular}{|c|p{10cm}|}
\hline
\ Parameter \  & Keterangan \\
\hline
$\Lambda$ & $\begin{matrix}\text{\ Jumlah individu yang masuk ke sistem}\end{matrix}$ \\
\hline
$\beta_1$ & $\begin{matrix}\text{\ Laju penularan melalui interaksi S-U}\end{matrix}$ \\
\hline
$\beta_2$ & $\begin{matrix}\text{\ Laju penularan melalui interaksi S-Rop}\end{matrix}$ \\
\hline
$\mu$ & $\begin{matrix}\text{\ Laju kematian alami}\end{matrix}$ \\
\hline
$\rho_1$ & $\begin{matrix}\text{\ Laju relapse (kambuh) dari Rop}\end{matrix}$ \\
\hline
$\rho_2$ & $\begin{matrix}\text{\ Laju relapse (kambuh) dari Rip}\end{matrix}$ \\
\hline
$P$ & $\begin{matrix}\text{\ Proporsi pengguna yang masuk rawat inap}\end{matrix}$ \\
\hline
$\alpha$ & $\begin{matrix}\text{\ Laju rekrutmen rehabilitasi dari U}\end{matrix}$ \\
\hline
$\gamma_1$& \ Laju transisi Rop ke Rip \\
\hline

$\gamma_2$ & \ Laju transisi Rip ke Rop \\
\hline
$\delta_1$ & $\begin{matrix}\text{\ Laju pemulihan dari Rop}\end{matrix}$ \\
\hline
$\delta_2$ & $\begin{matrix}\text{\ Laju pemulihan dari Rip}\end{matrix}$ \\
\hline
$\theta$ & $\begin{matrix}\text{\ Laju kematian U akibat narkoba}\end{matrix}$ \\
\hline


\end{tabular}
\end{center}
\end{table}
\endgroup
\subsection{Diagram Kompartemen}
\label{subsec:diagramkompartemen}
\begin{figure}[H]
    \centering
    \includegraphics[width=0.9\linewidth]{Untitled_Diagram.drawio__1_-removebg-preview (1).png}
    \caption{Diagram Model}
    \label{fig:enter-label}
\end{figure}


\subsection{Model}

\begin{align*}
\dfrac{dS}{dt} &= \Lambda - \beta_1 SU - \beta_2 S R_{op} - \mu S, \\
\dfrac{dU}{dt} &= \beta_1 SU + \beta_2 S R_{op} + \rho_1 R_{op} + \rho_2 R_{ip} - (\mu + \alpha + \theta)U, \\
\dfrac{dR_{op}}{dt} &= \gamma_2 R_{ip} - (\mu + \gamma_1 + \rho_1 + \delta_1)R_{op} + (1 - p)\alpha U, \\
\dfrac{dR_{ip}}{dt} &= \gamma_1 R_{op} - (\mu + \gamma_2 + \rho_2 + \delta_2)R_{ip} + p\alpha U.
\end{align*}


\subsection{ Nilai Parameter}

Nilai parameter yang ada pada pemodelan ini antara lain:

\begingroup
\allowdisplaybreaks
\begin{table}[H]
\begin{center}
%% increase table row spacing, adjust to taste
\renewcommand{\arraystretch}{1.125}
\caption{Tabel Parameter}
\centering
\begin{tabular}{|c|c|}
\hline
\ Parameter \  & Nilai \\
\hline
$\Lambda$ & $\begin{matrix}\text{\ 0.0209}\end{matrix}$ \\
\hline
$\beta_1$ & $\begin{matrix}\text{\ 0.0411}\end{matrix}$ \\
\hline
$\beta_2$ & $\begin{matrix}\text{\ 0.063}\end{matrix}$ \\
\hline
$\mu$ & $\begin{matrix}\text{\ 0.00749}\end{matrix}$ \\
\hline
$\rho_1$ & $\begin{matrix}\text{\ 0.0382}\end{matrix}$ \\
\hline
$\rho_2$ & $\begin{matrix}\text{\ 0.002}\end{matrix}$ \\
\hline
$P$ & $\begin{matrix}\text{\ 0.352}\end{matrix}$ \\
\hline
$\alpha$ & $\begin{matrix}\text{\ 0.02827}\end{matrix}$ \\
\hline
$\gamma_1$& \ 0.02961 \\
\hline

$\gamma_2$ & \ 0.003 \\
\hline
$\delta_1$ & $\begin{matrix}\text{\ 0.01}\end{matrix}$ \\
\hline
$\delta_2$ & $\begin{matrix}\text{\ 0.3142}\end{matrix}$ \\
\hline
$\theta$ & $\begin{matrix}\text{\ 0.005}\end{matrix}$ \\
\hline


\end{tabular}
\end{center}
\end{table}
\endgroup


\subsection{Simulasi Model}
tolong yang ada file simulasinya bantu masukin ya


\subsection{Titik Kesetimbangan Bebas Narkoba}
\begin{align*}
\frac{dS}{dt} &= \frac{dU}{dt} = \frac{dR_{op}}{dt} = \frac{dR_{ip}}{dt} = U = R_{op} = R_{ip} = 0
\end{align*}

% Garis panah menggunakan downarrow
\begin{center}
$\downarrow$
\end{center}

\begin{align*}
S &= \frac{\Lambda}{\mu} = 2.7903
\end{align*}

% Garis panah menggunakan downarrow
\begin{center}
$\downarrow$
\end{center}

\begin{align*}
(S, U, R_{op}, R_{sp}) &= (2.7903, 0, 0, 0)
\end{align*}

\subsection{Linearisasi Model}
\begin{equation*}
\dot{Y} = A^* Y
\end{equation*}

\begin{equation*}
\begin{pmatrix}
\dfrac{dS}{dt} \\[8pt]
\dfrac{dU}{dt} \\[8pt]
\dfrac{dR_{op}}{dt} \\[8pt]
\dfrac{dR_{tp}}{dt}
\end{pmatrix}
=
\begin{pmatrix}
-0.0075 & -0.1231 & -0.1758 & 0 \\
0 & 0.0823 & 0.2140 & 0.0020 \\
0 & 0.0183 & -0.0853 & 0.0030 \\
0 & 0.0099 & 0.0296 & -0.3267
\end{pmatrix}
\begin{pmatrix}
S \\[4pt]
U \\[4pt]
R_{op} \\[4pt]
R_{tp}
\end{pmatrix}
\end{equation*}

\subsection{Analisis Kestabilan Model Linear}
Analisis kestabilan model linear berdasarkan nilai eigen matriks \( A^* \) sebagai berikut:

\begin{equation*}
(\lambda_1, \lambda_2, \lambda_3, \lambda_4) = (-0.0075, 0.1032, -0.1059, -0.3270)
\end{equation*}

Karena terdapat nilai eigen positif (\( \lambda_2 = 0.1032 > 0 \)), maka:

\textbf{Model linear tidak stabil}